% Created 2025-08-30 sáb 22:22
% Intended LaTeX compiler: pdflatex
\documentclass[11pt]{article}
\usepackage[utf8]{inputenc}
\usepackage[T1]{fontenc}
\usepackage{graphicx}
\usepackage{longtable}
\usepackage{wrapfig}
\usepackage{rotating}
\usepackage[normalem]{ulem}
\usepackage{amsmath}
\usepackage{amssymb}
\usepackage{capt-of}
\usepackage{hyperref}
\date{\today}
\title{}
\hypersetup{
 pdfauthor={},
 pdftitle={},
 pdfkeywords={},
 pdfsubject={},
 pdfcreator={Emacs 30.2 (Org mode 9.7.11)}, 
 pdflang={English}}
\begin{document}

\tableofcontents

Concepção de Sistemas
Briefing
Profa. Márcia Zechlinski Gusmão
Perguntas
\begin{enumerate}
\item Qual o nome da empresa? Ziggurat
\item Qual o segmento da empresa? Segurança da informação
\item Qual o público-alvo da empresa? Pessoas interessadas em aumentar a segurança de suas credenciais por meio de um gerenciador de senhas com criptografia de ponta a ponta.

\item O projeto se refere a (1) site, (2) sistema, (3) aplicativo? Ou todos? Um sistema primeiramente web que pode contar com uma extensão para navegador.
\item A empresa possui marca ou Manual de Identidade Visual? Ainda não possui.
\item Qual o objetivo do projeto? O objetivo é desenvolver um gerenciador de senhas seguro, que utilize criptografia de ponta a ponta para armazenar credenciais em banco de dados local.

\item Quais são os 3 principais concorrentes da empresa? Bitwarden, KeePassXC e 1Password.
\item Quais são as 3 referências de sites modelo?KeePassXC, Bitwarden e Proton Pass.
\item Qual o prazo de lançamento do produto? O prazo de lançamento do protótipo está alinhado ao cronograma do TCC, com entrega da versão inicial funcional até o final do semestre letivo.
\item Vamos fazer um contrato? Sim. No contexto acadêmico, o projeto será documentado e validado como parte da disciplina de Trabalho de Conclusão de Curso (TCC).
\end{enumerate}
\end{document}
