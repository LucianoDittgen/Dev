% Created 2025-08-29 sex 22:26
% Intended LaTeX compiler: pdflatex
\documentclass[11pt]{article}
\usepackage[utf8]{inputenc}
\usepackage[T1]{fontenc}
\usepackage{graphicx}
\usepackage{longtable}
\usepackage{wrapfig}
\usepackage{rotating}
\usepackage[normalem]{ulem}
\usepackage{amsmath}
\usepackage{amssymb}
\usepackage{capt-of}
\usepackage{hyperref}
\date{\today}
\title{}
\hypersetup{
 pdfauthor={},
 pdftitle={},
 pdfkeywords={},
 pdfsubject={},
 pdfcreator={Emacs 30.2 (Org mode 9.7.11)}, 
 pdflang={English}}
\begin{document}

\tableofcontents

Para refletir:

\begin{enumerate}
\item Qual a importância de utilizar requisitos em um projeto de desenvolvimento de Software?
Para definir o que o sistema deve fazer, evita retrabalho e facilita os testes.
\item Diferencie Elicitação de Especificação de requisitos.
Na etapa de elicitação se coletam os requisitos e na especificação se documentam os mesmos.
\item Quais as principais técnicas de Elicitação de Requisitos? Descreva-as.Entrevista, questionários, observação, brainstorming, prototipação, workshops e análise de documentos.
\item Cite 5 Requisitos Funcionais para:
\end{enumerate}
a) Sistema da padaria de pequeno porte. 
Registro de vendas, gerar relatórios, homologar produtos, registrar pagamentos e ter controle do estoque.

b) Sistema de alocação docente. Cadastro dos professores, disciplinas, alocar de fato os docentes, gerar relatórios e editar as alocações.
\begin{enumerate}
\item Cite 5 Requisitos Não Funcionais para:
\end{enumerate}
a) Sistema da padaria de pequeno porte; 
b) Sistema de alocação docente.
\begin{enumerate}
\item Escreva um caso de uso chamado Manter Professor para o sistema de alocação docente.
\item Faça uma pesquisa mais aprofundada sobre o padrão arquitetural MVC, observe seus
\end{enumerate}
detalhes e formas de adaptação.
\begin{enumerate}
\item Cite três diferenças entre a fase de análise e a fase de projeto.
\item Qual a importância da fase de análise para um projeto de desenvolvimento de software?
\end{enumerate}


\end{document}
